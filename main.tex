\documentclass[titlepage, a4paper]{article}
%\usepackage{euler}
\usepackage{graphicx, amssymb, amsmath, textcomp, booktabs}
\usepackage[libertine,vvarbb]{newtxmath}
\usepackage[scr=rsfso]{mathalfa}
% \usepackage[lining,semibold,type1]{libertine} % a bit lighter than Times--no osf in math
\usepackage[T1]{fontenc} % best for Western European languages
\usepackage{minted}
\usepackage{listings, color, setspace, titlesec, fancyhdr, mdframed, multicol}
\usepackage{ucharclasses}
\usepackage{xunicode, xltxtra}
\usepackage[inner=1.35cm, outer=0.9cm, top=1.7cm, bottom=0.0cm]{geometry}
\usepackage{pdfpages}
\usepackage{tocloft}
\usepackage{nameref}
\usepackage{verbatim}
\usepackage{relsize}
\usepackage{fontspec}
\usepackage[colorlinks, linkcolor = black]{hyperref}
\usepackage[table]{xcolor}
\usepackage{tabularx}
% configure fonts
% if not using CJK
% \newfontfamily\substitutefont{SimSun}[Scale=0.8,BoldFont=SimHei]
% \setTransitionsForChinese{\begingroup\substitutefont}{\endgroup}
\usepackage{xeCJK}
\setCJKmainfont{Microsoft YaHei}[Scale=0.8]
\setCJKmonofont{SimHei}[Scale=0.8]
\setCJKsansfont{KaiTi}[Scale=0.8]

\setmainfont{Cambria}[Scale=0.925]
\setmonofont{Consolas}[Scale=0.775]
%\setsansfont{Gill Sans Medium}

\XeTeXlinebreaklocale "zh"
\XeTeXlinebreakskip = 0pt plus 1pt

\setlength{\parindent}{0em}\setlength{\parskip}{1pt}
\setlength\itemsep{1pt}

\makeatletter
\renewcommand{\paragraph}{%
	\@startsection{paragraph}{4}%
	{\z@}{1pt \@plus 1pt \@minus 1pt}{-1em}%
	{\normalfont\normalsize\bfseries}%
}
\makeatother


%configure the top corners
\pagestyle{fancy}
\setlength{\headsep}{0.1cm}

\chead{Acm template}
\rhead{Page \thepage}
\lhead{black\_trees}

%configure space between the two columns
\setlength{\columnsep}{13pt}

%configure minted to display codes
%\definecolor{Gray}{rgb}{0.9,0.9,0.9}

%remove leading numbers in table of contents
%\setcounter{secnumdepth}{0}	

%configure section style of table of content
\renewcommand\cftsecfont{\Large}

%configure section style
\titleformat{\section}
{\huge}			% The style of the section title
{\thesection.}				% a prefix
{4pt}						% How much space exists between the prefix and the title
{}					% How the section is represented
% \titleformat{\section}{\huge}{}{0pt}{}
\titlespacing{\section}{0pt}{0pt}{0pt}
\titlespacing{\subsection}{0pt}{0pt}{0pt}
\titlespacing{\subsubsection}{0pt}{0pt}{0pt}

%enable section to start new page automatically
%\let\stdsection\section
%\renewcommand\section{\penalty-100\vfilneg\stdsection}

%\renewcommand\theFancyVerbLine{\arabic{FancyVerbLine}}
\renewcommand{\theFancyVerbLine}{\sffamily \textcolor[rgb]{0.5,0.5,0.5}{\scriptsize {\arabic{FancyVerbLine}}}}

\setminted[cpp]{
	style=xcode,
	mathescape,
	linenos,
	autogobble,
	baselinestretch=0.8,
	tabsize=3,
	fontsize=\normalsize,
	%bgcolor=Gray,
	frame=single,
	framesep=1mm,
	framerule=0.3pt,
	numbersep=1mm,
	breaklines=true,
	breaksymbolsepleft=2pt,
	%breaksymbolleft=\raisebox{0.8ex}{ \small\reflectbox{\carriagereturn}}, %not moe!
	%breaksymbolright=\small\carriagereturn,
	breakbytoken=false,
	showtabs=true,
	tab={\relscale{0.6} $\big\vert \ \ \ $ \relscale{1}},
}
\setminted[java]{
	style=xcode,
	mathescape,
	linenos,
	autogobble,
	baselinestretch=0.8,
	tabsize=3,
	fontsize=\normalsize,
	%bgcolor=Gray,
	frame=single,
	framesep=1mm,
	framerule=0.3pt,
	numbersep=1mm,
	breaklines=true,
	breaksymbolsepleft=2pt,
	%breaksymbolleft=\raisebox{0.8ex}{ \small\reflectbox{\carriagereturn}}, %not moe!
	%breaksymbolright=\small\carriagereturn,
	breakbytoken=false,
	showtabs=true,
	tab={\relscale{0.6} $\big\vert \ \ \ $ \relscale{1}},
}
\setminted[python]{
	style=xcode,
	mathescape,
	linenos,
	autogobble,
	baselinestretch=0.8,
	tabsize=3,
	fontsize=\normalsize,
	%bgcolor=Gray,
	frame=single,
	framesep=1mm,
	framerule=0.3pt,
	numbersep=1mm,
	breaklines=true,
	breaksymbolsepleft=2pt,
	%breaksymbolleft=\raisebox{0.8ex}{ \small\reflectbox{\carriagereturn}}, %not moe!
	%breaksymbolright=\small\carriagereturn,
	breakbytoken=false,
	showtabs=true,
	tab={\relscale{0.6} $\big\vert \ \ \ $ \relscale{1}},
}

\setminted[vim]{
	style=xcode,
	mathescape,
	linenos,
	autogobble,
	baselinestretch=0.8,
	tabsize=2,
	fontsize=\normalsize,
	%bgcolor=Gray,
	frame=single,
	framesep=1mm,
	framerule=0.3pt,
	numbersep=1mm,
	breaksymbolsepleft=2pt,
	%breaksymbolleft=\raisebox{0.8ex}{ \small\reflectbox{\carriagereturn}}, %not moe!
	%breaksymbolright=\small\carriagereturn,
	breakbytoken=false,
}

\setminted[sh]{
	style=xcode,
	mathescape,
	linenos,
	autogobble,
	baselinestretch=0.8,
	tabsize=2,
	fontsize=\normalsize,
	%bgcolor=Gray,
	frame=single,
	framesep=1mm,
	framerule=0.3pt,
	numbersep=1mm,
	breaklines=true,
	breaksymbolsepleft=2pt,
	%breaksymbolleft=\raisebox{0.8ex}{ \small\reflectbox{\carriagereturn}}, %not moe!
	%breaksymbolright=\small\carriagereturn,
	breakbytoken=false,
}

%THE SCL BEGINS
\begin{document}
	\begin{titlepage}
		by black\_trees
	\end{titlepage}
	\begin{multicols}{2}
		\setcounter{tocdepth}{3}
		\begingroup
		\let\cleardoublepage\relax
		\let\clearpage\relax
		\begin{small}
			\begin{spacing}{0.75}
				\tableofcontents
			\end{spacing}
		\end{small}
		\newpage
		\begin{spacing}{0.6}
			\section{Tools}
				\subsection{vimrc}
				 	\inputminted{vim}{src/Tools/.vimrc}
				 \subsection{对拍}
				 	\inputminted{sh}{src/Tools/stress.sh}
				 \subsection{注意事项}
				 	\begin{itemize}
				 		\item Linux 下开大栈空间: 如果 ulimit -a 是 unlimited, 那么写入 ulimit -s 65536; ulimit -m 1048576 即可。
				 	\end{itemize}
			 \section{Ds}
			 	\subsection{链表}
			 		\inputminted{cpp}{src/Ds/List.cpp}
			 	\subsection{哈希表}
			 		\inputminted{cpp}{src/Ds/Hash_table.cpp}
			 	\subsection{并查集}
			 		\inputminted{cpp}{src/Ds/Dsu.cpp}
			 	\subsection{树状数组}
			 		\inputminted{cpp}{src/Ds/Fenwick_tree.cpp}
			 	\subsection{线段树}
			 		\inputminted{cpp}{src/Ds/Segment_tree.cpp}
			 	\subsection{轻重链剖分}
			 		\inputminted{cpp}{src/Ds/Hld.cpp}
			 	\subsection{主席树}
			 		\inputminted{cpp}{src/Ds/Persistent_seg.cpp}
			 	\subsection{李超树}
			 		\inputminted{cpp}{src/Ds/Li_chao_tree.cpp}
			 	\subsection{珂朵莉树}
			 		\inputminted{cpp}{src/Ds/Odt.cpp}
			 \section{Dp}
			 	\subsection{数位 DP}
			 		\inputminted{python}{src/Dp/Dight_dp.py}
			 	\subsection{子集卷积/SOS DP}
			 		\inputminted{cpp}{src/Dp/Sos_dp.cpp}
			 \section{Math}
			 	\subsection{线性求逆元以及组合数}
			 		\inputminted{cpp}{src/Math/Comb_inv.cpp}
			 	\subsection{快速幂}
			 		\inputminted{cpp}{src/Math/Qpow.cpp}
			 	\subsection{矩阵乘法}
			 		\inputminted{cpp}{src/Math/Matrix_mul.cpp}
			 	\subsection{高斯消元}
			 		\inputminted{cpp}{src/Math/Gauss_cut.cpp}
			 	\subsection{质数}
			 		\inputminted{cpp}{src/Math/Primes.cpp}
			 	\subsection{约数}
			 		\inputminted{cpp}{src/Math/Factors.cpp}
			 	\subsection{Exgcd}
			 		\inputminted{cpp}{src/Math/Ex_gcd.cpp}
			 	\subsection{中国剩余定理}
			 		\inputminted{cpp}{src/Math/Crt.cpp}
			 \section{Graph}
			 	\subsection{SPFA 以及负环}
			 		\inputminted{cpp}{src/Graph/Spfa.cpp}
			 	\subsection{Dijkstra}
			 		\inputminted{cpp}{src/Graph/Dijkstra.cpp}
			 	\subsection{Floyd 以及最小环}
			 		\inputminted{cpp}{src/Graph/Floyd.cpp}
			 	\subsection{Kruskal}
			 		\inputminted{cpp}{src/Graph/Kruskal.cpp}
			 	\subparagraph{Prim}
			 		\inputminted{cpp}{src/Graph/Prim.cpp}
			 	\subsection{倍增 LCA}
			 		\inputminted{cpp}{src/Graph/Multiper_lca.cpp}
			 	\subsection{Tarjan LCA}
			 		\inputminted{cpp}{src/Graph/Tarjan_Lca.cpp}
			 	\subsection{拓扑排序}
			 		\inputminted{cpp}{src/Graph/Topo_sort.cpp}
			 	\subsection{欧拉回路}
			 		\inputminted{cpp}{src/Graph/Euler_path.cpp}
			 	\subsection{强连通分量}
			 		\inputminted{cpp}{src/Graph/Tarjan_scc.cpp}
			 	\subsection{边双连通分量}
			 		\inputminted{cpp}{src/Graph/Tarjan_edcc.cpp}
			 	\subsection{点双连通分量}
			 		\inputminted{cpp}{src/Graph/Tarjan_vdcc.cpp}
			 	\subsection{虚树}
			 		\inputminted{cpp}{src/Graph/Virtual_tree.cpp}
			 \section{String}
			 	\subsection{Kmp}
			 		\inputminted{cpp}{src/String/Kmp.cpp}
			 	\subsection{Trie}
			 		\inputminted{cpp}{src/String/Trie.cpp}
			 	\subsection{01Trie}
			 		\inputminted{cpp}{src/String/01_trie.cpp}
			 	\subsection{Ac Automaton}
			 		\inputminted{cpp}{src/String/Ac_automaton.cpp}
		\end{spacing}
			\endgroup
	\end{multicols}
	
	\begin{center}
		\LARGE{Good Luck \&\& Have Fun!}
	\end{center}
		
	\end{document}
	%THE SCL ENDS
